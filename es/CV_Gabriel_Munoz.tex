%%%%%%%%%%%%%%%%%%%%%%%%%%%%%%%%%%%%%%%%%
% Twenty Seconds Resume/CV
% LaTeX Template
% Version 1.1 (8/1/17)
%
% This template has been downloaded from:
% http://www.LaTeXTemplates.com
%
% Original author:
% Carmine Spagnuolo (cspagnuolo@unisa.it) with major modifications by 
% Vel (vel@LaTeXTemplates.com)
%
% License:
% The MIT License (see included LICENSE file)
%
%%%%%%%%%%%%%%%%%%%%%%%%%%%%%%%%%%%%%%%%%

%----------------------------------------------------------------------------------------
%	PACKAGES AND OTHER DOCUMENT CONFIGURATIONS
%----------------------------------------------------------------------------------------

\documentclass[letterpaper]{twentysecondcv} % a4paper for A4
\usepackage[utf8]{inputenc}
\usepackage[T1]{fontenc}
\usepackage[spanish]{babel}
%----------------------------------------------------------------------------------------
%	 PERSONAL INFORMATION
%----------------------------------------------------------------------------------------

% If you don't need one or more of the below, just remove the content leaving the command, e.g. \cvnumberphone{}

\profilepic{gabriel.png} % Profile picture

\cvname{Gabriel Muñoz} % Your name
\cvjobtitle{Responsable de Tecnología} % Job title/career

\cvdate{15 de Julio de 1978} % Date of birth
\cvaddress{Sevilla, España} % Short address/location, use \newline if more than 1 line is required
\cvnumberphone{+34 660227709} % Phone number
\cvsite{linkedin.com/in/fgabrielmunoz} % Personal website
\cvmail{fgabriel.munoz@gmail.com} % Email address

%----------------------------------------------------------------------------------------

\begin{document}

%----------------------------------------------------------------------------------------
%	 ABOUT ME
%----------------------------------------------------------------------------------------

\aboutme{Más de 15 años de experiencia en el desarrollo de software, donde he desempeñado todos los roles: programador, analista, responsable de equipos y de proyectos, y responsable de tecnología.

Amplia experiencia en diferentes tecnologías, en la integración de sistemas complejos, en la convivencia de soluciones \emph{legacy} con software actual y en la renovación tecnológica de productos.} % To have no About Me section, just remove all the text and leave \aboutme{}

%----------------------------------------------------------------------------------------
%	 SKILLS
%----------------------------------------------------------------------------------------

% Skill bar section, each skill must have a value between 0 an 6 (float)
\skills{{CMMI/4.5},{Agile/5},{PMP/4},{Seguridad/4},{Java/6}, {Movilidad/5}, {Gestión de equipos/6}}

%------------------------------------------------

% Skill text section, each skill must have a value between 0 an 6
%\skillstext{{lovely/4},{narcissistic/3}}

%----------------------------------------------------------------------------------------

\makeprofile % Print the sidebar

%----------------------------------------------------------------------------------------
%	 INTERESTS
%----------------------------------------------------------------------------------------

\section{Intereses}

Interés en todos aquellos enfoques metodológicos que permitan aumentar la calidad y la predictibilidad de los procesos de desarrollo de software, tanto en metodologías clásicas (CMMI) como en metodologías ágiles (Scrum, Kanban), así como en la forma en que las metodologías clásicas conviven y refuerzan a las ágiles.
Interés en proyectos internacionales y/o relacionados con la investigación universitaria y la innovación tecnológica.

%----------------------------------------------------------------------------------------
%	 EDUCATION
%----------------------------------------------------------------------------------------

\section{Educación}

\begin{twenty} % Environment for a list with descriptions
\twentyitem{2012-2012}{Programa Ejecutivo en Dirección de Proyectos}{EOI}{Preparatorio para la obtención de la certificación PMP}	
	\twentyitem{2011-2012}{Máster en Administración y Dirección de empresas}{EOI}{Segundo Mejor Proyecto Fin de Máster de la promoción}	
	\twentyitem{2000-2002}{Ingeniería en Informática}{UGR}{Calificación media: Sobresaliente}
	\twentyitem{1996-1999}{Ingeniería Técnica en Informática de Gestión}{UCO}{Calificación media: Sobresaliente}
	%\twentyitem{<dates>}{<title>}{<location>}{<description>}
\end{twenty}


%----------------------------------------------------------------------------------------
%	 EXPERIENCE
%----------------------------------------------------------------------------------------

\section{Experiencia}

\begin{twenty} % Environment for a list with descriptions

	\twentyitem{2003 - Act.}{Responsable de Tecnología}{Aytos Berger Levrault}{He tenido diferentes etapas en los últimos años:
\begin{itemize}
\item Desde 2015, dirigiendo la renovación tecnológica, con el gran reto de transformar las ya exitosas soluciones actuales en el referente del futuro. Dirijo y coordino la estrategia tecnológica del grupo Berger Levrault en España. Participo también como referente técnico en proyectos internacionales del grupo.
\item En el periodo 2013-2015: dirigiendo el área de Administración Electrónica, Arquitectura e Integraciones. Con un importante equipo de ingenieros de desarrollo y certificación desarrollamos soluciones líderes en el ámbito de la Administración Local en Gestión Documental, BPM, Portales, Contratación Electrónica, Business Intelligence y soluciones móviles de acceso a servicios municipales por parte de los ciudadanos.
\end{itemize}
Con anterioridad asumí el puesto de responsable técnico de productos del área económico-financiera (Gestión Patrimonial, Gestión de Subvenciones), liderando los equipos de desarrollo y evolución técnica de los productos.}

	\twentyitem{2002-2003}{Analista-Programador}{Coritel - Accenture}{Participé fundamentalmente en dos proyectos:
\begin{itemize}
\item Análisis y desarrollo de la Intranet Corporativa de \emph{Cepsa}.
\item Desarrollo del Sistema de Aprovisionamiento de Recursos Humanos de \emph{Correos}.
\end{itemize}
}

	\twentyitem{1999-2000}{Programador}{Aconsa - CajaSur}{Participé como desarrollador en el proyecto de creación de un ERP a medida para \emph{CajaSur Renting}.}
	%\twentyitem{<dates>}{<title>}{<location>}{<description>}
\end{twenty}

%----------------------------------------------------------------------------------------
%	 OTHER INFORMATION
%----------------------------------------------------------------------------------------

\section{Otra información}
\subsection{Certificaciones}
\begin{itemize}
\item OCA Java 8
\item ITIL Foundation
\item CMMI Associate
\item Cambridge First Certificate in English (B2)
\end{itemize}
\subsection{Movilidad}
Carné de conducir tipo B y movilidad geográfica internacional.

%----------------------------------------------------------------------------------------
%	 SECOND PAGE EXAMPLE
%----------------------------------------------------------------------------------------

%\newpage % Start a new page

%\makeprofile % Print the sidebar

%\section{Other information}

%\subsection{Review}

%Alice approaches Wonderland as an anthropologist, but maintains a strong sense of noblesse oblige that comes with her class status. She has confidence in her social position, education, and the Victorian virtue of good manners. Alice has a feeling of entitlement, particularly when comparing herself to Mabel, whom she declares has a ``poky little house," and no toys. Additionally, she flaunts her limited information base with anyone who will listen and becomes increasingly obsessed with the importance of good manners as she deals with the rude creatures of Wonderland. Alice maintains a superior attitude and behaves with solicitous indulgence toward those she believes are less privileged.

%\section{Other information}

%\subsection{Review}

%Alice approaches Wonderland as an anthropologist, but maintains a strong sense of noblesse oblige that comes with her class status. She has confidence in her social position, education, and the Victorian virtue of good manners. Alice has a feeling of entitlement, particularly when comparing herself to Mabel, whom she declares has a ``poky little house," and no toys. Additionally, she flaunts her limited information base with anyone who will listen and becomes increasingly obsessed with the importance of good manners as she deals with the rude creatures of Wonderland. Alice maintains a superior attitude and behaves with solicitous indulgence toward those she believes are less privileged.

%----------------------------------------------------------------------------------------

\end{document} 
